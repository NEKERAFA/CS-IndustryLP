En este proyecto se ha enmarcado en el área de la representación del conocimiento. En especial, hemos construido una herramienta declarativa con la que elaborar, mediante programación lógica, polígonos industriales. Esto nos ha permitido crear una herramienta que, en manos de un experto, puede automatizar procesos de construcción de nuevos entornos industriales. Además, al apoyarnos en una herramienta de simulación como es \cities, podemos simular distintas carga de tráfico, una vez generado el polígono. \\

Como ya hemos comentado, gracias al uso de una herramienta declarativa, y de que la parte de definición del programa lógico es externo al algoritmo de búsqueda de modelos estables; un experto en construcción urbanística o de diseño de plantas industriales podría modificar o añadir reglas a su antojo para que la herramienta se adaptara lo máximo posible a su forma de trabajar. Además puede servir como base de datos en un futuro para guardar el tipo de estilo de expertos. \\

Otra parte importante de este trabajo es la construcción del binding de Clingo para el lenguaje de programación C\#. Aún no siendo el objetivo de este proyecto, al ser una biblioteca creada de forma desacoplada del proyecto principal, permite que más personas puedan usar las herramientas de Clingo, en este caso en proyectos creados para .NET. Esto permite una difusión mayor de la programación lógica, y en concreto de Answer Set Programming, para sistemas en los que posiblemente no se hubiera planteado hasta ahora como solución. \\

Para finalizar, este trabajo nos ha permitido ver como funciona un motor de videojuegos tan importante como lo es Unity, y en concreto como incorporar nuevas funcionalidades para un software que su fabricante no tenía pensado hasta ahora. Esto puede servir de gran ayuda para personas que quieran expandir su forma de jugar con un videojuego, además de servir de ejemplo y documentación para gente que esté pensando en entrar en el mundo del desarrollo de videojuegos.

\section{Trabajo futuro}

Con la finalización de este trabajo, se pueden sugerir diferentes líneas de trabajo futuro:

\begin{itemize}
	\item Permitir la generación de edificios de DLCs de \cities como \textit{Industries} o \textit{Sunset Harbor}, añadiendo las mecánicas de procesado de recursos. Además se podría incluir la generación de zonas de aparcamiento, parques y zonas verdes.
	\item Añadir más funcionalidades que nos permite clingo, como reglas de minimización y/o maximización, que permitan crear problemas de optimización a la hora de generar polígonos industriales. Además se podría añadir pesos a las reglas para permitir una generación más personalizada en función de lo que nos comente un experto sobre la construcción de polígonos industriales.
	\item Añadir reglas que permitan también la generación de las topologías de carreteras, pudiendo definir de forma declarativa las distribuciones e incluso unificando el proceso de generación.
	\item Mejorar la usabilidad de la herramienta, permitiendo marcar zonas a mano alzada o con distintas zonas y que tengan en cuenta el terreno. Además se podría incluir nuevas opciones para indicar, por ejemplo, la temática de los edificios a construir o prohibir la generación de un tipo concreto de edificio.
	\item Añadir información externa a la zona a construir, pudiendo unir de forma automática a autopistas o favoreciendo la conexión con zonas ya construidas.
\end{itemize}