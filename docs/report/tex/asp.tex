Este apéndice sirve como documentación y referencia de cada uno de los distintos ficheros con extensión \texttt{.lp} que forman en conjunto el programa lógico de IndustryLP. Debido a la naturaleza de la herramienta, estos ficheros se pueden modificar en caliente para modificar y/o restringir el espacio de búsqueda de modelos estables. Además se pueden añadir nuevos ficheros \texttt{.lp} que completen el programa lógico añadiéndolos a la carpeta detallada en la sección \ref{subsec:logic-files} del Apéndice \ref{manual}.

\subsubsection{\large\texttt{item\_definition.lp}}

En este fichero se definen las constantes que conformar nuestro sistema lógico.

\begin{lstlisting}[basicstyle=\ttfamily, numberstyle={\color[rgb]{0.25, 0.25, 0.25}\ttfamily}, numbers=left]
#const rows = 2.
#const columns = 2.

%%%%%%%%%%%%%%%%%%%%%%%%%%%%%%%%%%%%%%%
%%% Definimos los ambitos de las parcelas
%%%%%%%%%%%%%%%%%%%%%%%%%%%%%%%%%%%%%%%

% Fila de la cuadricula
row(0 .. (rows-1)).
% Columna de la cuadricula
column(0 .. (columns-1)).
\end{lstlisting}

\subsubsection{\large\texttt{generator.lp}}

En este fichero se especifican las reglas que permiten generar cada una de las distintas soluciones a nuestro problema.

\begin{lstlisting}[basicstyle=\ttfamily, numberstyle={\color[rgb]{0.25, 0.25, 0.25}\ttfamily}, numbers=left]
%%%%%%%%%%%%%%%%%%%%%%%%%%%%%%%%%%%%%%%
%%% Se generan las parcelas
%%%%%%%%%%%%%%%%%%%%%%%%%%%%%%%%%%%%%%%

0 { parcel(X, Y, S) : str_parcel(S) } 1 :- row(X), column(Y).

%%%%%%%%%%%%%%%%%%%%%%%%%%%%%%%%%%%%%%%
%%% Comprobaciones
%%%%%%%%%%%%%%%%%%%%%%%%%%%%%%%%%%%%%%%

sell_parcel(X, Y) :- row(X), column(Y), str_parcel(S), parcel(X, Y, S).

distance(S1, S2, |X1-X2|+|Y1-Y2|) :- parcel(X1, Y1, S1), parcel(X2, Y2, S2), X1 != X2, Y1 != Y2.

neighbour(S1, S2) :- distance(S1, S2, 1).

%%%%%%%%%%%%%%%%%%%%%%%%%%%%%%%%%%%%%%%
%%% Restricciones
%%%%%%%%%%%%%%%%%%%%%%%%%%%%%%%%%%%%%%%

:- parcel(X, Y, S1), parcel(X, Y, S2), str_parcel(S1), str_parcel(S2), row(X), column(Y), S1 != S2.

:- row(X), column(Y), not sell_parcel(X, Y).
\end{lstlisting}

\section{Reglas dinámicas}

Existen varias reglas que son añadidas al programa lógico en tiempo de ejecución. A continuación se especifica cada una de estas reglas.

\subsubsection{\large\texttt{str\_parcel(building\_name).}}

Antes de llamar al \textit{grounder} de Clingo, se obtienen los nombres de los \textit{prefabs} que corresponden con los edificios \textit{growables} que están cargados en \cities en ese momento y que son de tipo Industrial.

\subsubsection{\large\texttt{parcel(row, column, building\_name).}}

Cuando en la interfaz añadidos un edificio como preferencia, esto se traduce a un hecho lógico dentro del programa que se añade antes de ejecutar el \textit{ground} del programa lógico.

\subsubsection{\large\texttt{:- parcel(row, column, building\_name).}}

Si se añade un edificio como restricción, este se transformará en una prohibición dentro del programa lógico. Esto hará que no se busque soluciones en donde ese edificio aparezca en esa parcela.