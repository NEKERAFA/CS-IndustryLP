De un tiempo a esta parte, hemos visto cómo la industria de los videojuegos ha experimentado un cambio drástico, haciendo que actualmente sea uno de los  motores económicos en el entretenimiento a nivel mundial, teniendo casos en los que un proyecto de este campo supera en presupuesto, producción y facturación a muchas obras de la industria del cine. Esto se debe a todos los avances tecnológicos que nutren al mundo de los videojuegos, así como la madurez de un medio en el que muchos autores han visto su reconocimiento no por la diversión que plantean sus obras, sino por llevar al videojuego a su máxima expresión, realizándolo desde una forma más creativa o desarrollando el contenido y el alma del mismo de un modo que pueda llegar a un gran público. \\

Uno de los puntos que hacen crecer esta industria es la Inteligencia Artificial, ya que el poder que aporta esta rama de la informática para crear sistemas eficientes o que se comporten de una manera inteligente permite llevar al videojuego a un nuevo nivel. Estos sistemas permiten crear interacciones humano-máquina más naturales al aplicarlas a las distintas entidades que pueden conformar el universo de un videojuego, como enemigos inteligentes que aprenden la forma de jugar del usuario o entidades aliadas que ajustan su comportamiento a la experiencia del jugador. Además, permiten aplicar una variedad y diversidad al ecosistema al usar elementos de programación evolutiva a entidades \cite{5286468} o generación procedimental \cite{parkin_2016} para generar escenarios. \\

Incluso hemos visto el año pasado cómo el aprendizaje profundo o \textit{deep learning} ha permitido que empresas de \textit{hardware} como \textit{Nvidia} han construido su nueva generación de tarjetas gráficas \footnote{https://www.nvidia.com/es-es/geforce/graphics-cards/30-series} \footnote{https://developer.nvidia.com/rtx} para el mundo de los videojuegos con la premisa de un avance muy significativo en la calidad del renderizado de escenas generadas por computador, ya que permiten realizar una técnica muy costosa como puede ser el trazado de rayos o \textit{raytracing} \cite{Whitted:1980:IIM:358876.358882} en tiempo real \cite{Parker:2013:GRT:2447976.2447997}. \\

Volviendo a la generación procedimental de entornos, muchos de estos sistemas se basan en una generación pseudo-aleatoria de puntos en donde se crean distintos patrones ya definidos por un ser humano. Esto se repite, incluso aplicando transformaciones de estos patrones hasta generar un mapa que parezca real en un alto grado. El problema de estos sistemas llega al momento de crear un entorno grande, resultando extremadamente lentos, por lo que otra aproximación muy usada es usar funciones matemáticas que definen el contorno del terreno, pudiendo incluso generarlo infinitamente en tiempo real a medida que el usuario avanza. Estas aproximaciones realizan un gran trabajo en cuanto a eficiencia y rapidez, mas resultan ineficientes a la hora de plantear requisitos y modificaciones concretas, ya que muchos de los elementos que definen estos sistemas están expuestos a la incertidumbre debido a su propia construcción. Este proyecto se centra en la aplicación de otro tipo de aproximación para la resolución de este problema, empezando por un caso concreto de generación de entornos para un sistema de entretenimiento. \\

La idea central de este proyecto es la de usar estas técnicas de representación del conocimiento para automatizar parcialmente la planificación de parques industriales en un videojuego de simulación urbanística. Para ello se construirá una herramienta declarativa gobernada por reglas que genere el entramado de un polígono industrial y cuya salida pueda ser empleada por el simulador de ciudades de \cities, que permite crear, modificar y ejecutar una simulación de un ambiente urbano, y del que se pueden extraer características que se puedan contemplar parta la generación del parque, tales como información del tráfico, número de habitantes o zonas de viviendas cercanas, entre otras.

\section{Motivación}

Como ya se ha comentado anteriormente, muchos de los campos de la Inteligencia Artificial se están aplicando cada vez más en la industria de los videojuegos, en especial para facilitar la tarea de crear sistemas y entornos que sean orgánicos y naturales para el consumidor de este tipo de \textit{software}. Para ello se han aplicado muchas aproximaciones y optimizaciones de cara a tener sistemas lo más flexibles y con el objetivo de que respondan en un tiempo razonable consumiendo los mínimos recursos posibles. Este último requisito es imprescindible dado que un videojuego tiene que ser interactivo y con una tasa de respuesta lo más pequeña posible, así como poder ser ejecutado en sistemas con recursos muy escasos, como puede ser el caso de una consola portátil o un teléfono inteligente. \\

A pesar de esto, muchos de los sistemas generadores de escenarios presentan problemas en el momento de ser modificados, ya que para influir en el resultado de la generación se necesita hacer una reprogramación completa del algoritmo generado. Esto hace que una modificación pequeña de los parámetros internos puede llegar a ocasionar que el resultado varíe enormemente, haciendo incluso que en ciertos videojuegos que requieren de una conexión a Internet para ser ejecutados, tengan que reiniciar el escenario, haciendo que sus usuarios pierdan el progreso explorado en el mapa. Ejemplos de esto los tenemos en muchos servidores del videojuego \textit{Minecraft} o en las actualizaciones del videojuego \textit{No Man's Sky}. Debido a esto, muchas empresas prescinden de realizar actualizaciones a las partes del código que controlan la generación del universo, lastrando consigo problemas que pueden ocasionar fallos por la mala gestión que pudo ocurrir a la hora de diseñar el sistema. \\

Como propuesta para esta problemática, se desarrollará un generador que pueda crear un conjunto de entramados de carreteras dentro del videojuego \cities. En este entramado se colocarán elementos del juego tales como naves, empresas, zonas de aparcamiento y zonas verdes, además de establecer una conexión de esta zona con el resto de carreteras del entorno (conexiones con autovías para el transporte de mercancías y la conexión a las zonas residenciales), y la generación de conexiones con los servicios mínimos que necesitan los edificios (traída de agua y de residuos, y electricidad). \\

El generador a construir será gobernado por un conjunto de reglas, expresadas mediante restricciones lógicas y lineales representadas en un programa lógico. De este modo, el generador se puede adaptar con gran flexibilidad a distintos criterios de planificación urbanística, permitiendo que cualquier experto habituado al juego experimente modificaciones de dichos criterios mediante simples cambios en las restricciones del programa. Como ejemplo de prueba para estos criterios, se utilizarán reglas inspiradas en leyes y ordenanzas municipales reales \cite{guia_galicia} que regulen la forma del entramado y construcción del polígono en el propio juego, así como criterios propios del ámbito de arquitectura y planificación urbanística que pueden estar dados por un experto que tenga un gran conocimiento del simulador. \\

Para codificar las reglas que definen el problema, se utilizará el paradigma de programación lógica conocido como \asp \cite{asp}. Este paradigma se ha convertido hoy en día en uno de los lenguajes de representación de conocimiento con mayor proyección y difusión debido tanto a su eficiencia en aplicación práctica para la resolución de problemas como a su flexibilidad y expresividad para la representación del conocimiento. La generación de escenarios realistas \cite{5783900} supone un desafío como caso de prueba para \asp, ya que el número de combinaciones posibles aumenta exponencialmente en función del tamaño del escenario. \\

El proyecto estudiará distintas técnicas para reducir esta explosión combinatoria manteniendo en la medida de lo posible, el carácter declarativo de la herramienta. A su vez, el generador podrá obtener información de \cities sobre tráfico e información de los habitantes para poder generar la configuración más adecuada, ya sea incluyendo paradas de transporte público con las zonas residenciales, restricciones de horarios o tipos de vehículos que transitan las conexiones con el entorno. Cabe remarcar que ya existen antecedentes de generación declarativa para el diseño de espacios o entornos \cite{freeciv-editor} \cite{desing} usando este paradigma, así como el uso de \asp en otros ámbitos donde también ocurre el mismo problema combinatorio, como puede ser la composición musical \cite{haspie}\cite{DBLP:journals/corr/abs-1006-4948}.

\section{Objetivos}

Teniendo en cuenta lo explicado anteriormente, los objetivos finales de este proyecto son los siguientes:

\begin{itemize}
	\item Obtener un programa lógico (conjunto de predicados y reglas lógicas) que permita incorporar criterios y restricciones de diseño (simplificados) de un parque industrial.
	\item Permitir la entrada de información de un mapa de ciudad de \cities en forma de hechos para predicados del programa lógico.
	\item Permitir recuperar la salida obtenida por el programa lógico (en forma de hechos) y visualizarla como mapa de polígono industrial en \cities.
	\item Proporcionar una interfaz gráfica (ya sea mediante un editor 2D a medida o bien usando características del juego \cities) para marcar zonas de construcción o generar automáticamente restricciones lógicas que se quieran aplicar a las soluciones.
\end{itemize}

\section{Estructura de la memoria}

Para tener en cuenta la estructura que seguirá la memoria del presente proyecto, a continuación se explica los capítulos en los que se divide esta memoria:

\begin{itemize}
	\item \textbf{\hyperlink{introduccion}{Capítulo \ref*{introduction}. Introducción:}} Sirve como punto inicial a la lectura y conocimiento de este proyecto, describiendo la motivación que lo impulsa y detallando los objetivos a alcanzar.
	\item \textbf{\hyperlink{background}{Capítulo \ref*{background}. Contexto:}} Enmarca conceptos que son necesarios para entender el proyecto desarrollado, definiendo el plano tecnológico actual. Así mismo se describe y justifica las principales tecnologías empleadas en el desarrollo del mismo.
	\item \textbf{\hyperlink{mainwork}{Capítulo \ref*{mainwork}. Trabajo desarrollado:}} Explica las técnicas y el proceso de ingeniería llevado a cabo para la gestión, desarrollo y construcción del sistema propuesto para este proyecto. En este capítulo se indica el funcionamiento interno de la utilidad declarativa creada a la hora de plantear este proyecto.
	\item \textbf{\hyperlink{conclusions}{Capítulo \ref*{conclusions}. Conclusiones:}} Ofrece una visión global de la viabilidad y calidad del sistema obtenido, así como se indican las vías de trabajo futura que se abre a la finalización del mismo.
	\item \textbf{Apéndices:} Adjunta las siguientes secciones complementarias:
	\begin{itemize}
		\item \textbf{\hyperlink{manual}{A. Manual de usuario:}} Recoge la documentación para el uso de este proyecto por parte de un usuario.
		\item \textbf{B. Bibliografía:} Recoge la documentación bibliográfica sobre la que se apoya este proyecto.
	\end{itemize}
\end{itemize}

\newpage

\section{Plan de trabajo}

Para el desarrollo del proyecto se han seguido las siguientes etapas:

\begin{itemize}
	\item Estudio y análisis del estado del arte sobre la generación de escenarios en entornos similares, así como la investigación y estudio de la documentación de la API de desarrollo de plug-ins de \cities. 
	\item Análisis del estado del arte sobre generación de entornos urbanos.
	\item Diseño del programa lógico: eleccion de los predicados relevantes para el diseño del polígono indsutrial en un mapa de \cities, y las reglas que los relacionan.
	\item Diseño e implementación de un módulo que traduzca el estado del mapa de \cities en un momento concreto a hechos para predicados del programa en \asp.
	\item Diseño e implementación de un módulo del programa lógico que genere entramados de carreteras en \asp.
	\item Diseño e implementación de un plug-in (de tipo mod) de \cities que permita, durante la ejecución del juego, marcar zonas y/o restricciones, llamar a la herramienta de \asp para generar un polígono industrial y recuperar la salida de dicha herramienta mostrado el polígono ya construido en el juego.
	\item Evaluación de eficiencia para distintos casos de prueba creados a priori.
	\item Redacción de la memoria del proyecto final. Esta fase se ha intentado realizar de forma paralela al resto de etapas.
\end{itemize}