\documentclass[12pt, a4paper]{report}

\usepackage[utf8]{inputenc}
\usepackage[spanish, es-tabla]{babel}
\usepackage[dvips]{graphicx}
\usepackage{epsfig}
\usepackage[right]{eurosym}
\usepackage{geometry}
\usepackage{graphics}
\usepackage{tabularx}
\usepackage{svg}
\usepackage{adjustbox}
\usepackage{threeparttable}
\usepackage{amssymb}
\usepackage{listings}
\usepackage{amsmath}
\usepackage{pdfpages}
\usepackage{geometry}
\usepackage{fancyhdr}
\usepackage[hidelinks,pdftex,
pdfauthor={Rafael Alcalde Azpiazu},
pdftitle={Scenario Generation for a 2D Videogame using Logic Programming},
pdfsubject={Trabajo de Fin de Grado},
pdfkeywords={Answer Set Programming, Programación Lógica, Representación del Conocimiento, Resolución de problemas lógicos, Generación de mapas por ordenador, Freeciv, Lua},
pdfproducer={LaTeX},
pdfcreator={pdflatex}]{hyperref}

\renewcommand{\baselinestretch}{1.5}
\selectlanguage{spanish}

\begin{document}
\vspace*{4em}
{\large\bfseries{Generación de Escenarios de un Videojuego 2D mediante Programación Lógica}\par}
\vspace*{2em}
{\large\bfseries{Resumen}\par}
\vspace*{2em}
{Este proyecto consiste en el desarrollo de una herramienta que, mediante el paradigma de la programación lógica, permita generar entornos jugables para un videojuego 2D táctico. Estos escenarios serán propuestos mediante un modelo declarativo creado en \textit{Answer Set Programming}, el cual permite la representación del Conocimiento mediante lógica proposicional. Esto permitirá diseñar la construcción del entorno mediante una serie de reglas y restricciones, las cuales pueden ser modificadas fácilmente.\par}
\vspace*{\fill}
\end{document}