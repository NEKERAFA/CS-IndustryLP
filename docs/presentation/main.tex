\documentclass[10pt]{beamer}

% ---- Packages ----

\usepackage[utf8]{inputenc}
\usepackage[T1]{fontenc}
\usepackage[spanish]{babel}
\usepackage{epsfig}
\usepackage{graphics}
\usepackage{tabularx}
\usepackage{listings}
\usepackage{float}
\usepackage{multirow}
\usepackage{noto}
\usepackage{eurosym}
\usepackage{epsfig}
\usepackage{amssymb}
\usepackage{amsmath}
\usepackage{xcolor}

% ---- Configuration ----

\usetheme{Madrid}
\useinnertheme{circles}
\usenavigationsymbolstemplate{}
\setbeamertemplate{blocks}[rounded]

\definecolor{UDCpink}{RGB}{214,30,140}
\definecolor{UDCgray}{RGB}{100,100,100}

\usecolortheme[named=UDCpink]{structure}

\makeatletter
\setbeamertemplate{title page}
{
	\vbox{}
	\vspace{0.25em}
	\begin{centering}
		{\usebeamercolor[fg]{titlegraphic}\inserttitlegraphic\par}
		\vspace{1em}
		\begin{beamercolorbox}[sep=8pt,center,rounded=true]{title}
			\usebeamerfont{title}\inserttitle\par%
			\ifx\insertsubtitle\@empty%
			\else%
			\vspace{0.25em}
			{\usebeamerfont{subtitle}\usebeamercolor[fg]{subtitle}\insertsubtitle\par}%
			\fi%     
		\end{beamercolorbox}%
		\vfill
		\begin{beamercolorbox}[sep=8pt,center]{author}
			\usebeamerfont{author}\insertauthor
		\end{beamercolorbox}
		\begin{beamercolorbox}[sep=8pt,center]{institute}
			\usebeamerfont{institute}\insertinstitute
		\end{beamercolorbox}
		\begin{beamercolorbox}[sep=8pt,right]{date}
			\itshape\usebeamerfont{date}\insertdate
		\end{beamercolorbox}
	\end{centering}
	\vspace{0.25em}
}
\makeatother

\lstdefinestyle{myLuastyle}
{
	language         = {[5.2]Lua},
	commentstyle=\color{green},
	keywordstyle=\color{blue}, 
	basicstyle = \ttfamily \color{black} \footnotesize
}

\lstset{style=myLuastyle}

% ---- Document ----

\title[Generación mediante ASP]{Generación de Escenarios de un Videojuego 2D mediante Programación Lógica}
\author[Rafael Alcalde Azpiazu]{\large Rafael Alcalde Azpiazu}
\institute[UDC] % (optional)
{
	\normalsize
	Grado en Ingeniería Informática \\
	Mención en Computación \\
	\vspace{1em}
	Proyecto clásico de Ingeniería \\
	Facultad de Informática \\
	\vspace{1em}
	Director: Pedro Cabalar \\
}
\date[]{A Coruña, \today}
\titlegraphic{\includegraphics[height=2em]{images/logo.png}}

\begin{document}
	
	\begin{frame}
		\titlepage
	\end{frame}

	\section{Motivación}
	\begin{frame}
	\frametitle{Motivación}

	\begin{itemize}
		\item<1-> La \textcolor{UDCpink}{industria del videojuego} constituye un sector económico cada vez más relevante.
		
		\vspace{0.5em}
		
		\pause
		
		Beneficios netos (en dólares):
		
		\begin{itemize}
			\item Minecraft: 2500 millones, Fornite: 1000 millones, Destiny: 500 millones.
		\end{itemize}
	
		\vspace{1em}
		
		\item<3-> Ha activado avances tecnológicos, p. ej. el uso de la \textcolor{UDCpink}{Inteligencia Artificial}, uno de los campos que más ha contribuido.
		
		\vspace{0.5em}
		
		\begin{itemize}
			\item<4-> Diseño de \textcolor{UDCpink}{enemigos inteligentes}, p. ej mediante programación evolutiva (No Man's Sky).
			
			\vspace{0.5em}
			
			\item<5-> \textcolor{UDCpink}{Diseño del entorno}. Existen dos aproximaciones:
			
			\vspace{0.5em}
			
			\begin{itemize}
				\item<5->\small Generación procedimental. La más usada.
				
				\vspace{0.5em}
				
				\item<6->\small \textcolor{UDCpink}{Generación declarativa} $\Leftarrow$.
			\end{itemize}
		\end{itemize}
	\end{itemize}
\end{frame}

\begin{frame}
	\frametitle{Diseño de entornos}
	
	\begin{itemize}
		\item \textcolor{UDCpink}{Generación procedimental}: se basa en un algoritmo o técnica ya predefinida.
		
		\vspace{1em}
		
		\begin{enumerate}
			\item Algoritmo ad-hoc.
			
			\vspace{0.5em}
			
			\item Programación evolutiva.
			
			\vspace{0.5em}
			
			\item Expresiones matemáticas.
		\end{enumerate}
	\end{itemize}
	
	\vspace{1em}
	
	\pause
	
	\begin{block}{\normalsize Problemática}
		\vspace{1em}
		Para \textcolor{UDCpink}{influir} en el resultado de la generación se necesita \textcolor{UDCpink}{reprogramar} el algoritmo generador para adaptarlo a los criterios.
		\vspace{1em}
	\end{block}
\end{frame}

\begin{frame}
\frametitle{Diseño de entornos}

	\begin{itemize}
		\item<1-> \textcolor{UDCpink}{Generación declarativa}: existe una representación formal del entorno, p. ej. mediante \textcolor{UDCpink}{programación lógica}.
		
		\vspace{1em}
		
		\item<2-> La generación es \textcolor{UDCpink}{independiente} del algoritmo de búsqueda usado para obtener las posibles soluciones.
		
		\vspace{1em}
		
		\item<3-> Un caso concreto de programación lógica es \textcolor{UDCpink}{\textit{Answer Set Programming}}.
		
		\vspace{0.5em}
		
		\begin{itemize}
			\item<4-> \textit{Answer Set Programming for Procedural Content Generation: A Design Space Approach} \textcolor{UDCpink}{[Smith et al, 11]} (ASP + Warzone 2100).
			
			\vspace{0.5em}
			
			\item<5-> En este proyecto: \textcolor{UDCpink}{ASP + Freeciv} $\Leftarrow$.
		\end{itemize}
	\end{itemize}

\end{frame}
	\begin{frame}
	\frametitle{Freeciv}

	\begin{columns}
		
		\column{0.6\textwidth}
		\begin{itemize}
			\item<1-> Versión \textcolor{UDCpink}{\textit{open source}} y \textcolor{UDCpink}{gratuita} de Sid Meier's Civilization creado en la universidad de Aarhus.
			
			\vspace{1em}
			
			\item<2-> Juego de \textcolor{UDCpink}{estrategia por turnos}.
			
			\vspace{1em}
			
			\item<3-> El jugador controla a un \textcolor{UDCpink}{grupo de colonos}, comienza en el año \textcolor{UDCpink}{4000 A.C}.
			
			\vspace{1em}
			
			\item<4-> El objetivo final es crear una \textcolor{UDCpink}{gran civilización}. Para ello existen \textcolor{UDCpink}{5 formas} de finalizar el juego:
			
			\vspace{0.5em}
			
			\begin{itemize}
				\item Victoria por \textcolor{UDCpink}{dominación}, \textcolor{UDCpink}{científica}, \textcolor{UDCpink}{religión}, \textcolor{UDCpink}{cultural} o \textcolor{UDCpink}{por puntuación}.
			\end{itemize}
		\end{itemize}

		\column{0.4\textwidth}
		\includegraphics[width=\textwidth]{images/freeciv-example.png}
	\end{columns}

\end{frame}

\begin{frame}
\frametitle{Tipos de terrenos en Freeciv}

\begin{itemize}
	\item Hay \textcolor{UDCpink}{12 tipos de terreno}, con posibles bonificaciones.
\end{itemize}

\vspace{0.5em}

\centering
\includegraphics[width=0.8\textwidth]{images/ejemplo-partida.png}
\end{frame}


\begin{frame}
\frametitle{Objetivos del proyecto}

Para este proyecto:

\vspace{1em}

\begin{itemize}
	\item<1-> Se definirá un \textcolor{UDCpink}{modelo declarativo} del escenario para Freeciv usando \textcolor{UDCpink}{\itshape Answer Set Programming}.
	
	\vspace{1em}
	
	\item<2-> Se construirá una pequeña \textcolor{UDCpink}{herramienta gráfica} con la que manipular el escenario.
	
	\vspace{1em}
	
	\item<3-> Eficiencia: reducir o podar el número de combinaciones posibles.
\end{itemize}

\end{frame}

	
	\begin{frame}
	\frametitle{Índice}
		\tableofcontents
	\end{frame}
	
	\AtBeginSection[]
	{
	\begin{frame}
	\frametitle{Índice}
	\tableofcontents[currentsection]
	\end{frame}
	}
	
	\section{Answer Set Programming}
	Este apéndice sirve como documentación y referencia de cada uno de los distintos ficheros con extensión \texttt{.lp} que forman en conjunto el programa lógico de IndustryLP. Debido a la naturaleza de la herramienta, estos ficheros se pueden modificar en caliente para modificar y/o restringir el espacio de búsqueda de modelos estables. Además se pueden añadir nuevos ficheros \texttt{.lp} que completen el programa lógico añadiéndolos a la carpeta detallada en la sección \ref{subsec:logic-files} del Apéndice \ref{manual}.

\subsubsection{\large\texttt{item\_definition.lp}}

En este fichero se definen las constantes que conformar nuestro sistema lógico.

\begin{lstlisting}[basicstyle=\ttfamily, numberstyle={\color[rgb]{0.25, 0.25, 0.25}\ttfamily}, numbers=left]
#const rows = 2.
#const columns = 2.

%%%%%%%%%%%%%%%%%%%%%%%%%%%%%%%%%%%%%%%
%%% Definimos los ambitos de las parcelas
%%%%%%%%%%%%%%%%%%%%%%%%%%%%%%%%%%%%%%%

% Fila de la cuadricula
row(0 .. (rows-1)).
% Columna de la cuadricula
column(0 .. (columns-1)).
\end{lstlisting}

\subsubsection{\large\texttt{generator.lp}}

En este fichero se especifican las reglas que permiten generar cada una de las distintas soluciones a nuestro problema.

\begin{lstlisting}[basicstyle=\ttfamily, numberstyle={\color[rgb]{0.25, 0.25, 0.25}\ttfamily}, numbers=left]
%%%%%%%%%%%%%%%%%%%%%%%%%%%%%%%%%%%%%%%
%%% Se generan las parcelas
%%%%%%%%%%%%%%%%%%%%%%%%%%%%%%%%%%%%%%%

0 { parcel(X, Y, S) : str_parcel(S) } 1 :- row(X), column(Y).

%%%%%%%%%%%%%%%%%%%%%%%%%%%%%%%%%%%%%%%
%%% Comprobaciones
%%%%%%%%%%%%%%%%%%%%%%%%%%%%%%%%%%%%%%%

sell_parcel(X, Y) :- row(X), column(Y), str_parcel(S), parcel(X, Y, S).

distance(S1, S2, |X1-X2|+|Y1-Y2|) :- parcel(X1, Y1, S1), parcel(X2, Y2, S2), X1 != X2, Y1 != Y2.

neighbour(S1, S2) :- distance(S1, S2, 1).

%%%%%%%%%%%%%%%%%%%%%%%%%%%%%%%%%%%%%%%
%%% Restricciones
%%%%%%%%%%%%%%%%%%%%%%%%%%%%%%%%%%%%%%%

:- parcel(X, Y, S1), parcel(X, Y, S2), str_parcel(S1), str_parcel(S2), row(X), column(Y), S1 != S2.

:- row(X), column(Y), not sell_parcel(X, Y).
\end{lstlisting}

\section{Reglas dinámicas}

Existen varias reglas que son añadidas al programa lógico en tiempo de ejecución. A continuación se especifica cada una de estas reglas.

\subsubsection{\large\texttt{str\_parcel(building\_name).}}

Antes de llamar al \textit{grounder} de Clingo, se obtienen los nombres de los \textit{prefabs} que corresponden con los edificios \textit{growables} que están cargados en \cities en ese momento y que son de tipo Industrial.

\subsubsection{\large\texttt{parcel(row, column, building\_name).}}

Cuando en la interfaz añadidos un edificio como preferencia, esto se traduce a un hecho lógico dentro del programa que se añade antes de ejecutar el \textit{ground} del programa lógico.

\subsubsection{\large\texttt{:- parcel(row, column, building\_name).}}

Si se añade un edificio como restricción, este se transformará en una prohibición dentro del programa lógico. Esto hará que no se busque soluciones en donde ese edificio aparezca en esa parcela.
	
	\section{Demostración}
	
	\section{Trabajo desarrollado}
	\begin{frame}
\frametitle{Arquitectura del sistema}

\begin{itemize}
	\item<1-> \textcolor{UDCpink}{\itshape Front-end}: Creado en \textcolor{UDCpink}{LÖVE}, usa un modelo MVC que usa una interfaz gráfica en modo inmediato.
	
	\vspace{0.5em}
	
	\item<2-> \textcolor{UDCpink}{\itshape Back-end}: Creado en \textcolor{UDCpink}{Lua} (con API de clingo) y \textcolor{UDCpink}{ASP}, usa un modelo en \textit{pipeline}.
\end{itemize}

\pause[3]

\begin{center}
	\includegraphics[width=.95\textwidth]{images/arquitectura-completa.pdf}
\end{center}

\end{frame}

\begin{frame}
\frametitle{Proceso de ingeniería}

\begin{itemize}
	\item<1-> Metodología: \textcolor{UDCpink}{desarrollo iterativo incremental y evolutivo}.
	
	\vspace{0.5em}
	
	\item<2-> Se ha usado herramientas de \textcolor{UDCpink}{uso libre y gratuitas} para el desarrollo del proyecto.
	
	\vspace{0.5em}
	
	\item<3-> El coste total del proyecto asciende a \textcolor{UDCpink}{\EUR{3720.00}}.
	
	\vspace{0.5em}	
\end{itemize}

\pause[4]

\centering
\includegraphics[width=\textwidth]{images/gantt.pdf}

\end{frame}

	\section{Evaluación}
	En este capítulo se analizará los resultados obtenidos en el proceso de evaluación del sistema. Para la planificación de las pruebas se ha optado por el uso de una herramienta de integración continua en donde se corren tanto pruebas de unidad a las partes del modelo de la interfaz gráfica, así como a los distintos módulos de los programas lógicos. \\

Además se ha usado un programa escrito en C\# para la ejecución de pruebas de rendimiento o \textit{benchmarks} el cual solo usa el módulo clingo para evitar el uso de la interfaz gráfica. En todas estas pruebas se ha añadido a las pruebas la restricción de que el cómputo del mapa solo puede durar como máximo cinco minutos, matando el proceso en caso de superar este tiempo y pasando a la siguiente prueba. Con esto podemos indicar cuales son las configuraciones que dan una experiencia pobre al usuario. \\

A continuación se describen los resultados obtenidos de estas pruebas de rendimiento con distintos parámetros. \\

\section{Tamaño de los mapas}

TODO

	\section{Conclusiones}
	\begin{frame}
\frametitle{Conclusiones}

\begin{itemize}
	\item El sistema permite definir \textcolor{UDCpink}{nuevas propiedades} de forma sencilla.
	
	\vspace{1em}
	
	\begin{itemize}
		\item Esto proporciona \textcolor{UDCpink}{flexibilidad} a la hora de adaptar el sistema.
		
		\vspace{0.5em}
		
		\item No hay que tener en cuenta el \textcolor{UDCpink}{método de resolución}.
	\end{itemize}

	\vspace{1em}

	\item Se ha reducido el problema de eficiencia ocasionado por el \textcolor{UDCpink}{número exponencial de combinaciones posibles}.
\end{itemize}

\end{frame}

\begin{frame}
\frametitle{Trabajo futuro}

\begin{itemize}
	\item Añadir otros elementos del mapa al generador (ríos, comida, animales, etc...).
	
	\vspace{1em}
	
	\item Mejorar el rendimiento del sistema.
	
	\vspace{1em}
	
	\item Posible mejora en la interfaz gráfica.
	
	\vspace{1em}
	
	\begin{itemize}
		\item Mejorar la manipulación del mapa.
		
		\vspace{0.5em}
		
		\item Añadir nuevos tipos de restricciones.
	\end{itemize}
	
	\vspace{1em}
	
	\item Publicar la herramienta para tener una base de usuarios.
\end{itemize}
\end{frame}

	\begin{frame}
		\vbox{}
		\vspace{0.25em}
		\begin{centering}
			{\usebeamercolor[fg]{titlegraphic}\inserttitlegraphic\par}
			\vspace{0.5em}
			\begin{beamercolorbox}[sep=8pt,center,rounded=true]{title}
				\usebeamerfont{title}\inserttitle\par%
				\ifx\insertsubtitle\@empty%
				\else%
				\vspace{0.25em}
				{\usebeamerfont{subtitle}\usebeamercolor[fg]{subtitle}\insertsubtitle\par}%
				\fi%     
			\end{beamercolorbox}%
			\vfill
			\begin{beamercolorbox}[sep=8pt,center]{author}
				\usebeamerfont{author}\insertauthor
			\end{beamercolorbox}
			\vspace{0.5em}
			{\Large\textcolor{UDCpink}{¡Gracias por su atención!}\par}
			\vspace{0.5em}
			\begin{beamercolorbox}[sep=8pt,center]{institute}
				\usebeamerfont{institute}\insertinstitute
			\end{beamercolorbox}
			\begin{beamercolorbox}[sep=8pt,right]{date}
				\itshape\usebeamerfont{date}\insertdate
			\end{beamercolorbox}
		\end{centering}
		\vspace{0.25em}
	\end{frame}

	\begin{frame}
	\frametitle{Ejemplo de la biblioteca de Clingo}
	
	\begin{block}{Añadir restricciones al programa}
		\ttfamily \footnotesize
		
		\#script(lua)
		
		\textbf{function} main(prog)
		
		\hspace{2em} \textit{\textendash\textendash Genero las regiones}
		
		\hspace{2em} \textbf{for} i = 0, c\_regions-1 \textbf{do}
		
		\hspace{4em} \textit{\textendash\textendash Hace grounding del programa lógico}
		
		\hspace{4em} prog:ground(\{\{''base'', \{\}\}, \{''generate'', \{i\}\}\})
		
		\hspace{4em} \textit{\textendash\textendash Obtengo un manejador de la solución}
		
		\hspace{4em} handle = prog:solve({yield=true})
		
		\vspace{1em}
		
		\hspace{4em} \textbf{local} restrictions = '' ''
		
		\hspace{4em} \textit{\textendash\textendash Recorre los modelos de la solución}
		
		\hspace{4em} \textbf{for} model \textbf{in} handle:iter() \textbf{do}
		
		\hspace{6em} \textit{\textendash\textendash Añado las restricciones}
		
		\hspace{6em} \textbf{if} \#restrictions \textasciitilde{}= 0 \textbf{then}
		
		\hspace{8em} prog1:load(''resources/restrictions.lp'')
		
		\hspace{6em} \textbf{end}
		
		\hspace{6em} ...
		
	\end{block}
	\end{frame}
	
	\begin{frame}
	\frametitle{Ejemplo de la biblioteca de Clingo}
	
	\begin{block}{Añadir restricciones al programa}
	\ttfamily \footnotesize
	
	\hspace{6em} \textbf{for} m \textbf{in} handle:iter() \textbf{do}
	
	\hspace{8em} \textbf{for} row\_str, col\_str, contain \textbf{in} string.gmatch(tostring(model), 	''cell\%(p\%((\%d+),(\%d+)\%),(\%l+)\%)'') \textbf{do}
	
	\hspace{10em} \textbf{if} contain == "l"  \textbf{then}
	
	\hspace{12em} lands = lands .. '' land(p(''..row\_str..'',''..col\_str..'')).''
	
	\hspace{12em} restrictions = restrictions .. check\_restrictions(row, col, i)
	
	\hspace{10em} \textbf{end}
	
	\hspace{8em} \textbf{end}
	
	\vspace{1em}
	
	\hspace{8em} df = io.open(''resources/restrictions.lp'', ''w+'')
	
	\hspace{8em} df:write(restrictions)
	
	\hspace{8em} df:flush()
	
	\hspace{8em} df:close()
	
	\hspace{6em} \textbf{end}
	
	\hspace{4em} \textbf{end}
	
	\hspace{2em} \textbf{end}
	
	\textbf{end}
	
	\#end.
	\end{block}
	\end{frame}

	\begin{frame}
	\frametitle{Ejecución del módulo Clingo}
	
	\centering
	\includegraphics[width=0.6\textwidth]{images/diagrama-secuencia.pdf}
	
	\end{frame}

	\begin{frame}
	\frametitle{Ejemplo de generación 20x20}
	
	\centering
	\includegraphics[width=0.6\textwidth]{images/ejemplo-20-20.png}
	
	\end{frame}

	\begin{frame}
	\frametitle{Ejemplo de generación 30x30}
	
	\centering
	\includegraphics[width=0.6\textwidth]{images/ejemplo-30-30.png}
	
	\end{frame}

\end{document}