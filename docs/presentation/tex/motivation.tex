\begin{frame}
	\frametitle{Motivación}

	\begin{itemize}
		\item<1-> La \textcolor{UDCpink}{industria del videojuego} constituye un sector económico cada vez más relevante.
		
		\vspace{0.5em}
		
		\pause
		
		Beneficios netos (en dólares):
		
		\begin{itemize}
			\item Minecraft: 2500 millones, Fornite: 1000 millones, Destiny: 500 millones.
		\end{itemize}
	
		\vspace{1em}
		
		\item<3-> Ha activado avances tecnológicos, p. ej. el uso de la \textcolor{UDCpink}{Inteligencia Artificial}, uno de los campos que más ha contribuido.
		
		\vspace{0.5em}
		
		\begin{itemize}
			\item<4-> Diseño de \textcolor{UDCpink}{enemigos inteligentes}, p. ej mediante programación evolutiva (No Man's Sky).
			
			\vspace{0.5em}
			
			\item<5-> \textcolor{UDCpink}{Diseño del entorno}. Existen dos aproximaciones:
			
			\vspace{0.5em}
			
			\begin{itemize}
				\item<5->\small Generación procedimental. La más usada.
				
				\vspace{0.5em}
				
				\item<6->\small \textcolor{UDCpink}{Generación declarativa} $\Leftarrow$.
			\end{itemize}
		\end{itemize}
	\end{itemize}
\end{frame}

\begin{frame}
	\frametitle{Diseño de entornos}
	
	\begin{itemize}
		\item \textcolor{UDCpink}{Generación procedimental}: se basa en un algoritmo o técnica ya predefinida.
		
		\vspace{1em}
		
		\begin{enumerate}
			\item Algoritmo ad-hoc.
			
			\vspace{0.5em}
			
			\item Programación evolutiva.
			
			\vspace{0.5em}
			
			\item Expresiones matemáticas.
		\end{enumerate}
	\end{itemize}
	
	\vspace{1em}
	
	\pause
	
	\begin{block}{\normalsize Problemática}
		\vspace{1em}
		Para \textcolor{UDCpink}{influir} en el resultado de la generación se necesita \textcolor{UDCpink}{reprogramar} el algoritmo generador para adaptarlo a los criterios.
		\vspace{1em}
	\end{block}
\end{frame}

\begin{frame}
\frametitle{Diseño de entornos}

	\begin{itemize}
		\item<1-> \textcolor{UDCpink}{Generación declarativa}: existe una representación formal del entorno, p. ej. mediante \textcolor{UDCpink}{programación lógica}.
		
		\vspace{1em}
		
		\item<2-> La generación es \textcolor{UDCpink}{independiente} del algoritmo de búsqueda usado para obtener las posibles soluciones.
		
		\vspace{1em}
		
		\item<3-> Un caso concreto de programación lógica es \textcolor{UDCpink}{\textit{Answer Set Programming}}.
		
		\vspace{0.5em}
		
		\begin{itemize}
			\item<4-> \textit{Answer Set Programming for Procedural Content Generation: A Design Space Approach} \textcolor{UDCpink}{[Smith et al, 11]} (ASP + Warzone 2100).
			
			\vspace{0.5em}
			
			\item<5-> En este proyecto: \textcolor{UDCpink}{ASP + Freeciv} $\Leftarrow$.
		\end{itemize}
	\end{itemize}

\end{frame}